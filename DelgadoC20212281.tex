
\documentclass[letterpaper,11pt]{article}
%%%%%%%%%%%%%%%%%%%%%%%%%%%%%%%%%%%%%%%%%%%%%%%%%%%%%%%%%%%%%%%%%%%%%%%%%%%%%%%%%%%%%%%%%%%%%%%%%%%%%%%%%%%%%%%%%%%%%%%%%%%%%%%%%%%%%%%%%%%%%%%%%%%%%%%%%%%%%%%%%%%%%%%%%%%%%%%%%%%%%%%%%%%%%%%%%%%%%%%%%%%%%%%%%%%%%%%%%%%%%%%%%%%%%%%%%%%%%%%%%%%%%%%%%%%%
\usepackage{graphicx}
\usepackage{amsmath,amsfonts,amssymb,amsthm,float}
\usepackage{hyperref}
\usepackage[utf8]{inputenc}
\usepackage[left=2cm, right=2cm, top=2cm, bottom=2cm]{geometry}

\setcounter{MaxMatrixCols}{10}
%TCIDATA{OutputFilter=LATEX.DLL}
%TCIDATA{Version=5.50.0.2953}
%TCIDATA{<META NAME="SaveForMode" CONTENT="1">}
%TCIDATA{BibliographyScheme=BibTeX}
%TCIDATA{LastRevised=Monday, May 19, 2025 15:25:52}
%TCIDATA{<META NAME="GraphicsSave" CONTENT="32">}
%TCIDATA{ComputeDefs=
%$h_{4}=\alpha x$
%}


\input{tcilatex}
\renewcommand{\baselinestretch}{1.15}
\setlength{\parindent}{0pt}
\setlength{\parskip}{0.5\baselineskip}
\pretolerance=2000 \tolerance=3000
\renewcommand{\abstractname}{Resumen}

\begin{document}

\title{Analisis de sistemas biologicos}
\author{Delgado Soto Jose Sebastian $\left[ C20212281\right] $ \\
%EndAName
Departamento de Ingenier\'{\i}a El\'{e}ctrica y Electr\'{o}nica\\
Tecnol\'{o}gico Nacional de M\'{e}xico / Instituto Tecnol\'{o}gico de Tijuana%
}
\maketitle

\noindent \textbf{Palabras clave: }\ Modelo; 3 poblaciones; Sistema;
Tratamiento; Celulas.

\noindent

\noindent Correo: \textbf{l20212281@tectijuana.edu.mx}

\noindent \noindent Carrera: \textbf{Ingenier\'{\i}a Biom\'{e}dica }

\noindent Asignatura: \textbf{Gemelos Digitales}

\noindent Profesor: \href{https://biomath.xyz/}{\textbf{Dr. Paul Antonio
Valle Trujillo}} (paul.valle@tectijuana.edu.mx)

\section{Modelo Matematico}

El modelo matematico se compone por las siguientes tres Ecuaciones
Diferenciales Ordinarias (EDOs) de primer orde:

\begin{eqnarray*}
\dot{x} &=&r_{1}x\left( 1-b_{1}x\right) -a_{12}xy-a_{13}xz, \\
\dot{y} &=&r_{2}y\left( 1-b_{2}y\right) -a_{21}xy, \\
\dot{z} &=&\left( r_{3}-a_{31}\right) xz-d_{3}z+\rho _{i},
\end{eqnarray*}

donde $x\left( t\right) $ es la poblacion de celulas anormales, $y\left(
t\right) $ la poblacion de celulas normales y $z\left( t\right) $ la
poblacion de celulas efectoras, ademas el tiempo $t$ \ \ 

se mide en dias.

\textbf{\bigskip\ Comentarios sobre el modelo:}

\begin{enumerate}
\item Es un sistema de ecuaciones diferenciales ordinario de primer orden

\item Interacciona entre celulas anormales, normales y celulas efectoras

\item En un sistema de 3 poblaciones.

\item La poblacion de \ celulas anormales y normales se describe a travez de
la ley de crecimiento logistico

\item La celulas efectoras se describe mediante la accion de masas.

\item Los terminos $xy$ representan la competencia de recursos entre celulas
patologicas y sanas

\item La eliminacion de celulas patologicas por celulas efectoras t
supresion inmune es representada por el termino $xz$

\item $\rho i$ representa aplicacion externa de un tratamiento de
inmunoterapia.

\item La dinamica del sistema es de la forma presa-depredador de Lotka
Volterra

\item Debido a que el sistema describe la concentracion de poblaciones
celulares \ respecto al tiempo, sus soluciones deben ser no negativas para
condiciones iniciales no negativas, de lo contrario se perdera el
significado de sistema biologico
\end{enumerate}

\bigskip

\bigskip En esta seccion se aplica el lema de positividad para sistemas
dinamicos no lineales, por lo que se realizan la siguientes evaluaciones:

\begin{eqnarray*}
\left. \dot{x}\right\vert _{x=0} &=&r_{1}\left( 0\right) \left(
1-b_{1}\left( 0\right) \right) -a_{12}\left( 0\right) y-a_{13}\left(
0\right) z=0 \\
\left. \dot{y}\right\vert _{y=0} &=&r_{2}\left( 0\right) \left(
1-b_{2}\left( 0\right) \right) -a_{21}x\left( 0\right) ,=0, \\
\left. \dot{z}\right\vert _{z=0} &=&\left( r_{3}-a_{31}\right) x\left(
0\right) -d_{3}\left( 0\right) +\rho _{i}=\rho _{i},
\end{eqnarray*}

\bigskip

Se analizaron las ecuaciones cuando las ponlaciones valen $0$, comprobando
que el sistema no genera numeros negativos. Si las poblaciones valen 0, las
celulas anormales y normales no tienen un crecimimiento, sin embargo las
celulas efectoras

\bigskip

\bigskip Por lo tanto, de acuerdo con De leenheer \& Aeyels[1], se concluye
el siguiente resultado

\textbf{Resultado I. Positividad:} \textit{Las soluciones }$\left[ x\left(
t\right) ,y\left( t\right) ,z\left( t\right) \right] $\textit{\ y
semi-trayectorias positivas }$\left( \Gamma ^{+}\right) $\textit{\ del
sistema (ref:dx)-(ref: dz) seran positivamente invariantes y para cada
condicion inicial no negativa }$\left[ x\left( 0\right) ,y\left( 0\right)
,z\left( 0\right) \geq 0\right] $\textit{\ se localizaran en el siguiente
dominio:}

\begin{equation*}
R_{0}^{3}=\{x\left( t\right) ,y\left( t\right) ,z\left( t\right) \geq 0\}
\end{equation*}

\bigskip

\bigskip

\bigskip

\textbf{Referencia:}

\begin{itemize}
\item 
\begin{description}
\item De Leenher, P., \& Aeyels, D. (2001). Stability properties of
equilibria of classes of cooperative systems.IEEE Transactions on Automatic
Control,46(12), 1996-2001. https://doi.org/ 10.1109/9.975508\newline
\end{description}
\end{itemize}

\bigskip

\bigskip

\subsection{Localizacion de conjuntos compactos invariantes}

Primero, se debe de proponer una funcion localizadora, para sistemas
biologicos con dinamica localizada en el ortante no negativo, se sugiere
explorar las siguientes funciones

\begin{eqnarray*}
h_{1} &=&x, \\
h_{2} &=&y, \\
h_{3} &=&z, \\
h_{4} &=&x+y+z, \\
h_{5} &=&x+z \\
h_{6} &=&x,+y \\
h_{7} &=&y+z
\end{eqnarray*}

\bigskip

Nota: Con base en la estructura del sistema, se observa que las variables $%
x(t)$y $y(t)$, tienen los siguientes limites inferiores y superiores.

\bigskip

\begin{eqnarray*}
0 &\preceq &x\left( t\right) \preceq 1 \\
0 &\preceq &x\left( t\right) \preceq 1
\end{eqnarray*}

\bigskip

Esto corresponde con la ley de crecimiento logistico(crecimiento de tipo
sigmoidal), que tiende a cero al menos infinito y a uno hacia el infinito.

Se explota la siguiente funcion localizadora:

\begin{equation*}
h_{1}=x,
\end{equation*}

\bigskip

y se calcula su derivada de Lie(derivada temporal o derivada implicita con
respecto al tiempo):

\bigskip

\begin{equation*}
L_{f}h_{1}=\frac{dx}{dt}=\dot{x}=r_{1}x\left( 1-b_{1}x\right)
-a_{12}xy-a_{13}xz,
\end{equation*}

con lo cual, se formula el conjunto $S\left( h_{1}\right) =\{L_{f}h_{1}=0\}$%
, es decir,

\bigskip

\begin{equation*}
S\left( h_{1}\right) =\{r_{1}x\left( 1-b_{1}x\right) -a_{12}xy-a_{13}xz=0\}
\end{equation*}

se observa que este conjunto puede reescribirse de la siguiente forma:

\bigskip

\begin{equation*}
S\left( h_{1}\right) =\{r_{1}x\left( 1-b_{1}x\right)
-a_{12}xy-a_{13}xz=0\}\cup \{x=0\},
\end{equation*}

\bigskip

ahora, se reescribe la primera parte del conjunto, despejando la variable de
interes:

\begin{equation*}
S\left( h_{1}\right) =\{x=\frac{1}{b_{1}}-\frac{a_{12}}{r_{1}b_{1}}y-\frac{%
a_{13}}{r_{1}b_{1}}x\}\cup \{x=0\},
\end{equation*}

\bigskip

con base en los anterior se concluye lo siguiente:

\bigskip 
\begin{equation*}
K\left( h_{1}\right) =\{x_{\inf }=0\preceq x_{\max }=\frac{1}{b_{1}}\},
\end{equation*}

\bigskip

es decir, el valor minimo que puede tener la solucion $x(t)$ $\ $es de cero,
mientras que, el valor maximo que puede alcanzar esta solucion cuando $y=z=0$%
, es de uno (recordado que el sistema esta normalizado).

\bigskip

\bigskip \textbf{Para y}

Se explota la siguiente funcion localizadora:

\begin{equation*}
h_{2}=y,
\end{equation*}

\bigskip y se calcula su deribada derivada de Lie:

\begin{equation*}
L_{f}h_{2}=r_{2}y\left( 1-b_{2}y\right) -a_{21}xy,
\end{equation*}

Entonces, el conjunto $S\left( h_{2}\right) =\{L_{f}h_{2}=0\}$,esta dado por
lo siguiente

\begin{equation*}
S\left( h_{2}\right) =\{y=\frac{1}{b_{2}}-\frac{a_{12}}{r_{2}b_{2}}y-\frac{%
a_{13}}{r_{2}b_{2}}x\}\cup \{y=0\}
\end{equation*}

\bigskip

con base a lo anterior , se concluye el siguiente resultado:

\bigskip

\begin{equation*}
K\left( h_{2}\right) =\{y_{\inf }=0\preceq y_{\sup }=\frac{1}{b_{2}}\},
\end{equation*}

\bigskip

\bigskip Se explota la siguiente funcion localizadora:

\begin{equation*}
h_{3}=z,
\end{equation*}

\bigskip

al calcular su derivada de Lie

\bigskip

\begin{equation*}
L_{f}h_{3}=\left( r_{3}-a_{31}\right) xz-d_{3}z+\rho _{i},
\end{equation*}

Entonces, el conjunto $S\left( h_{3}\right) $,esta dado por lo siguiente

\bigskip 
\begin{equation*}
S\left( h_{3}\right) =\{L_{f}h_{3}=0\}=\{\left( r_{3}-a_{31}\right)
xz-d_{3}z+\rho _{i}=0\},
\end{equation*}

\bigskip

donde, al observar los valores de los parametros, se construte la siguiente
condicion:

\begin{equation*}
r_{3}>a_{31}
\end{equation*}

\bigskip

por lo tanto, se reescribe el conjunto $S\left( h_{3}\right) $ de la
siguiente forma:

\bigskip

\begin{equation*}
S\left( h_{3}\right) =\{z=\frac{\rho _{i}}{d_{3}}-\frac{r_{3}-a_{13}}{d_{3}}%
xz\},
\end{equation*}

\bigskip

por lo tanto, se observa que, la solucion tiene el siguiente limite inferior:

\bigskip

\begin{equation*}
K\left( z\right) =\{z\left( t\right) \geq z_{\inf }\frac{\rho _{i}}{d_{3}}\}
\end{equation*}

\bigskip

recordando que $p_{i}$ es el parametro de tratamiento / terapia (o parametro
de control), que puede tener valores no negativos, es decir $pi\geq 0.$

\bigskip

por lo tanto, con base en el reusltado anterior, se procede a aplicar el
denominado Teorema Iterativo del metodo de LCCI , entonces, se reescribe el
conjunto $S\left( h_{1}\right) $ como se muestra a continuacion:

\bigskip

\begin{eqnarray*}
S\left( h_{1}\right) &=&\{r_{1}x\left( 1-b_{1}x\right)
-a_{12}xy-a_{13}xz=0\}\cup \{x=0\},\text{(esto no se escribe de manera
formal)} \\
S\left( h_{1}\right) \cap K\left( z\right) &\subset &\{x=\frac{1}{b_{1}}-%
\frac{a_{12}}{r_{1}b_{1}}y-\frac{a_{13}}{r_{1}b_{1}}z_{\inf }\}
\end{eqnarray*}

\bigskip

ahora, al descartar el termino negativo de $y,$se concluye el siguiente
limite superior para la variable $x\left( t\right) $:

\bigskip

\begin{equation*}
K_{x}=\{x_{\inf }=0\leq x(t)\leq x_{\sup }=\frac{1}{b_{1}}-\frac{a_{13}}{%
r_{1}b_{1}d_{3}}pi
\end{equation*}

\bigskip

Finalmente, se toma la siguiente funcion localizadora:

\begin{equation*}
h_{4}=\alpha x+z,
\end{equation*}

\bigskip

\bigskip cuya derivada de Lie se muestra a continuacion:

\bigskip

\begin{equation*}
L_{f}h_{4}=\alpha \left[ r_{1}x\left( 1-b_{1}x\right) -a_{12}xy-a_{13}xz%
\right] +(r_{3}-a_{31})xz-d_{3}z+pi
\end{equation*}

\bigskip

se determina el conjunto $S\left( h_{4}\right) =\{L_{f}h_{4}=0\}$ de la
siguiente forma:

\begin{eqnarray*}
S\left( h_{4}\right) &=&\{\alpha r_{1}x-b_{1}\alpha r_{1}x^{2}-\alpha
a_{12}xy-\alpha _{13}xz+\left( r_{3}-a_{31}\right) xz-d_{3}z+pi=0\}, \\
S\left( h_{4}\right) &=&\{pi-b_{1}\alpha r_{1}x^{2}+ar_{1}x-\alpha
a_{12}xy-(\alpha a_{13}-r_{3}+a_{31})xz+\left( r_{3}-a_{31}\right)
xz-d_{3}z=0
\end{eqnarray*}

\bigskip

\bigskip

para asegurar que todos los terminos cruzados/ no lineales/cuadrativos, sean
negativos, se impone la siguiente condicion

\bigskip

\bigskip 
\begin{eqnarray*}
\alpha a_{13}-r_{3}+a_{31} &>&0 \\
\alpha &>&\frac{r_{3-}a_{31}}{a_{13}}
\end{eqnarray*}

\bigskip

ahora, la funcion localizaodra se puede expresar de esta forma:

\begin{equation*}
z=h_{4}-\alpha x,
\end{equation*}

\bigskip para sustituir en la siguiente expresion:

\begin{equation*}
S\left( h_{4}\right) =\left\{ d_{3}z=\rho _{i}-b_{1}\alpha r_{1}x^{2}+\alpha
r_{1}x-\alpha a_{12}xy-\left( \alpha a_{13}-r_{3}+a_{31}\right) xz\right\} ,
\end{equation*}%
es decir,

\begin{eqnarray*}
S\left( h_{4}\right) &=&\left\{ d_{3}\left( h_{4}-\alpha x\right) =\rho
_{i}-b_{1}\alpha r_{1}x^{2}+\alpha r_{1}x-\alpha a_{12}xy-\left( \alpha
a_{13}-r_{3}+a_{31}\right) xz\right\} , \\
S\left( h_{4}\right) &=&\left\{ d_{3}h_{4}=\rho _{i}-b_{1}\alpha
r_{1}x^{2}+\left( \alpha r_{1}+d_{3}\alpha \right) x-\alpha a_{12}xy-\left(
\alpha a_{13}-r_{3}+a_{31}\right) xz\right\} , \\
S\left( h_{4}\right) &=&\left\{ h_{4}=\frac{\rho _{i}}{d_{3}}-\frac{%
b_{1}\alpha r_{1}}{d_{3}}x^{2}+\frac{\alpha r_{1}+d_{3}\alpha }{d_{3}}x-%
\frac{\alpha a_{12}}{d_{3}}xy-\frac{\alpha a_{13}-r_{3}+a_{31}}{d_{3}}%
xz\right\} ,
\end{eqnarray*}%
para continuar con el proceso, primero se debe completar el cuadrado con los
siguientes dos terminos:

\begin{equation*}
-\frac{b_{1}\alpha r_{1}}{d_{3}}x^{2}+\frac{\alpha r_{1}+d_{3}\alpha }{d_{3}}%
x=-Ax^{2}+Bx=-A\left( x-\frac{B}{2A}\right) ^{2}+\frac{B^{2}}{4A}
\end{equation*}%
y se sustituye en el conjunto $S\left( h_{4}\right) $

\begin{equation*}
S\left( h_{4}\right) =\left\{ h_{4}=\frac{\rho _{i}}{d_{3}}+\frac{B^{2}}{4A}%
-A\left( x-\frac{B}{2A}\right) ^{2}-\frac{\alpha a_{12}}{d_{3}}xy-\frac{%
\alpha a_{13}-r_{3}+a_{31}}{d_{3}}xz\right\} ,
\end{equation*}%
por lo tanto, se concluye el siguiente limite superior para la funcion $%
h_{4} $:

\begin{equation*}
K\left( h_{4}\right) =\left\{ ax\left( t\right) +z\left( t\right) \leq \frac{%
\rho _{i}}{d_{3}}+\frac{\alpha \left( d_{3}+r_{1}\right) ^{2}}{%
4b_{1}d_{3}r_{1}}\right\} ,
\end{equation*}%
y se aproxima el siguiente limite superior para la variable $z\left(
t\right) $:

\begin{equation*}
K_{z}=\left\{ z_{\inf }=\frac{\rho _{i}}{d_{3}}\leq z\left( t\right) \leq
z_{\sup }=\frac{\rho _{i}}{d_{3}}+\frac{\alpha \left( d_{3}+r_{1}\right) ^{2}%
}{4b_{1}d_{3}r_{1}}\right\} .
\end{equation*}

Con base en lo mostrado en esta seccion, se concluye el siguiente resultado:

\bigskip

\textbf{Resultado II: Dominio de localizacion}\textit{: Todos los conjuntos
compactos invariantes del sistema }$\left( \ref{dx}\right) -\left( \ref{dz}%
\right) $\textit{\ se encuentran localizados dentro o en las fronteras del
siguiente dominio de localizacion:}

\bigskip 
\begin{equation*}
K_{xyz}=K_{x}\cap K_{y}\cap K_{z},
\end{equation*}%
\textit{donde}

\bigskip 
\begin{eqnarray*}
K_{x} &=&\left\{ x_{\inf }=0\leq x\left( t\right) \leq x_{\sup }=\frac{1}{%
b_{1}}-\frac{a_{13}}{r_{1}b_{1}d_{3}}\rho _{i}\right\} , \\
K_{y} &=&\left\{ y_{\inf }=0\leq y\left( t\right) \leq y_{\sup }=\frac{1}{%
b_{2}}\right\} , \\
K_{z} &=&\left\{ z_{\inf }=\frac{\rho _{i}}{d_{3}}\leq z\left( t\right) \leq
z_{\sup }=\frac{\rho _{i}}{d_{3}}+\frac{\alpha \left( d_{3}+r_{1}\right) ^{2}%
}{4b_{1}d_{3}r_{1}}\right\} .
\end{eqnarray*}

\bigskip

\bigskip

\subsection{No existencia de conjuntos compactos invariables}

\bigskip

A partir del resultado mostrado en el conjunto en el conjunto $Kx$, es
posible establecer lo siguiente con respecto a la existencia de conjuntos
compactos invariantes para la variables $x(t):$

\bigskip

\textbf{Resultado III: No existencia. }\textit{Si la siguiente condicion
sobre el parametro de tratamiento /terapia se cumple:}

\bigskip

\begin{equation*}
\frac{1}{b_{1}}-\frac{a_{13}}{r_{1}b_{1}d_{3}}pi\preceq 0,
\end{equation*}

es decir,

\bigskip

\begin{equation*}
pi\geq \frac{r_{1}d_{3}}{a_{13}},
\end{equation*}

entonces, se puede asegurar la no existencia de conjuntos compactos
invariantes fuera \ del plano $x=o$, por lo tanto, cualquier dinamica que
pueda exhibir el sistema, estara localizada dentro o fuera

\bigskip 
\begin{equation*}
K_{xyz}=\left\{ x=0\right\} \cap K_{y}\cap K_{z},
\end{equation*}

\bigskip

\subsection{Puntos de equilibrio}

\bigskip

Para calcular los puntos de equilibrio del sistema(ref:x)- (ref: dz), se
igualan a cero de las ecuaciones como se muestra a continuacion

\bigskip

$\func{assume}\left( r_{1},\func{positive}\right) =\allowbreak \left(
0,\infty \right) $

$\func{assume}\left( b_{1},\func{positive}\right) =\allowbreak \left(
0,\infty \right) $

$\func{assume}\left( a_{12},\func{positive}\right) =\allowbreak \left(
0,\infty \right) $

$\func{assume}\left( a_{13},\func{positive}\right) =\allowbreak \left(
0,\infty \right) $

$\func{assume}\left( r_{2},\func{positive}\right) =\allowbreak \left(
0,\infty \right) $

$\func{assume}\left( a_{21},\func{positive}\right) =\allowbreak \left(
0,\infty \right) $

$\func{assume}\left( r_{3},\func{positive}\right) =\allowbreak \left(
0,\infty \right) $

$\func{assume}\left( a_{31},\func{positive}\right) =\allowbreak \left(
0,\infty \right) $

$\func{assume}\left( d_{3},\func{positive}\right) =\allowbreak \left(
0,\infty \right) $

\bigskip primero, se calculan los equilibrios asumiendo $\rho i=0:$%
\begin{eqnarray*}
0 &=&r_{1}\left( 1-b_{1}x\right) -a_{12}xy-a_{13}xz \\
0 &=&r_{2}y\left( 1-b_{2}y\right) -a_{21}xy \\
0 &=&\left( r_{3}-a_{31}\right) xz-d_{3}z
\end{eqnarray*}

$\bigskip $

$\bigskip $%
\begin{eqnarray*}
&&\left[ x=\frac{d_{3}}{r_{3}-a_{31}},y=0,z=-\frac{1}{%
r_{3}a_{13}-a_{13}a_{31}}\left(
-r_{1}r_{3}+r_{1}a_{31}+b_{1}d_{3}r_{1}\right) \right] \\
&&\left[ x=\frac{d_{3}}{r_{3}-a_{31}},y=-\frac{1}{%
b_{2}r_{2}r_{3}-b_{2}r_{2}a_{31}}\left(
d_{3}a_{21}-r_{2}r_{3}+r_{2}a_{31}\right) ,z=\frac{1}{%
b_{2}r_{2}r_{3}a_{13}-b_{2}r_{2}a_{13}a_{31}}\left(
d_{3}a_{12}a_{21}-r_{2}r_{3}a_{12}+r_{2}a_{12}a_{31}+b_{2}r_{1}r_{2}r_{3}-b_{2}r_{1}r_{2}a_{31}-b_{1}b_{2}d_{3}r_{1}r_{2}\right) %
\right] \\
&&\left[ x=\frac{1}{a_{21}}\left( r_{2}-b_{2}r_{2}\frac{%
r_{1}a_{21}-b_{1}r_{1}r_{2}}{a_{12}a_{21}-b_{1}b_{2}r_{1}r_{2}}\right) ,y=%
\frac{r_{1}a_{21}-b_{1}r_{1}r_{2}}{a_{12}a_{21}-b_{1}b_{2}r_{1}r_{2}},z=0%
\right] \\
&&\left[ x=0,y=\frac{1}{b_{2}},z=0\right]
\end{eqnarray*}

\bigskip $\func{assume}\left( \rho _{i},\func{positive}\right) =\allowbreak
\left( 0,\infty \right) $

Ahora, considerando $\rho _{i}>0:$

\begin{eqnarray*}
0 &=&r_{1}x(1-b_{1}x)-a_{12}xy-a_{13}xz \\
0 &=&r_{2}y(1-b_{2}y)-a_{21}xy \\
0 &=&(r_{3}-a_{31})xz-d_{3}z
\end{eqnarray*}%
, Solution is: $\left\{ 
\begin{array}{ccc}
\left\{ \left[ y=0,z=0\right] ,\left[ x=\frac{d_{3}}{r_{3}-a_{31}},y=0,z=-%
\frac{1}{r_{3}a_{13}-a_{13}a_{31}}\left(
-r_{1}r_{3}+r_{1}a_{31}+b_{1}d_{3}r_{1}\right) \right] ,\left[ x=\frac{d_{3}%
}{r_{3}-a_{31}},y=-\frac{1}{b_{2}r_{2}r_{3}-b_{2}r_{2}a_{31}}\left(
d_{3}a_{21}-r_{2}r_{3}+r_{2}a_{31}\right) ,z=\frac{1}{%
b_{2}r_{2}r_{3}a_{13}-b_{2}r_{2}a_{13}a_{31}}\left(
d_{3}a_{12}a_{21}-r_{2}r_{3}a_{12}+r_{2}a_{12}a_{31}+b_{2}r_{1}r_{2}r_{3}-b_{2}r_{1}r_{2}a_{31}-b_{1}b_{2}d_{3}r_{1}r_{2}\right) %
\right] ,\left[ x=\frac{1}{a_{21}}\left( r_{2}-b_{2}r_{2}\frac{%
r_{1}a_{21}-b_{1}r_{1}r_{2}}{a_{12}a_{21}-b_{1}b_{2}r_{1}r_{2}}\right) ,y=%
\frac{r_{1}a_{21}-b_{1}r_{1}r_{2}}{a_{12}a_{21}-b_{1}b_{2}r_{1}r_{2}},z=0%
\right] ,\left[ x=0,y=\frac{1}{b_{2}},z=0\right] \right\} & \text{if} & 
b_{2}\neq 0\wedge -r_{3}+a_{31}\neq 0\wedge b_{2}^{2}\frac{r_{2}^{2}}{%
a_{21}^{2}}-\frac{1}{b_{1}}\frac{b_{2}}{r_{1}}r_{2}\frac{a_{12}}{a_{21}}\neq
0\wedge \frac{r_{1}a_{21}-b_{1}r_{1}r_{2}}{a_{12}a_{21}-b_{1}b_{2}r_{1}r_{2}}%
\neq 0 \\ 
\left\{ \left[ y=0,z=0\right] ,\left[ x=\frac{d_{3}}{r_{3}-a_{31}},y=0,z=-%
\frac{1}{r_{3}a_{13}-a_{13}a_{31}}\left(
-r_{1}r_{3}+r_{1}a_{31}+b_{1}d_{3}r_{1}\right) \right] ,\left[ x=\frac{d_{3}%
}{r_{3}-a_{31}},y=-\frac{1}{b_{2}r_{2}r_{3}-b_{2}r_{2}a_{31}}\left(
d_{3}a_{21}-r_{2}r_{3}+r_{2}a_{31}\right) ,z=\frac{1}{%
b_{2}r_{2}r_{3}a_{13}-b_{2}r_{2}a_{13}a_{31}}\left(
d_{3}a_{12}a_{21}-r_{2}r_{3}a_{12}+r_{2}a_{12}a_{31}+b_{2}r_{1}r_{2}r_{3}-b_{2}r_{1}r_{2}a_{31}-b_{1}b_{2}d_{3}r_{1}r_{2}\right) %
\right] ,\left[ x=0,y=\frac{1}{b_{2}},z=0\right] \right\} & \text{if} & 
b_{2}\neq 0\wedge -r_{3}+a_{31}\neq 0\wedge b_{2}^{2}\frac{r_{2}^{2}}{%
a_{21}^{2}}-\frac{1}{b_{1}}\frac{b_{2}}{r_{1}}r_{2}\frac{a_{12}}{a_{21}}\neq
0\wedge \frac{r_{1}a_{21}-b_{1}r_{1}r_{2}}{a_{12}a_{21}-b_{1}b_{2}r_{1}r_{2}}%
=0%
\end{array}%
\right. \allowbreak $

\bigskip

\bigskip

\begin{eqnarray*}
\dot{x} &=&r_{1}x(1-b_{1}x)-a_{12}xy-a_{13}xz, \\
\dot{y} &=&r_{2}y(1-b_{2}y)-a_{21}xy, \\
\dot{z} &=&(r_{3}-a_{31})xz-d_{3}z+\rho _{i},
\end{eqnarray*}

$\allowbreak $

\subsection{\protect\bigskip Condiciones de eliminacion}

\bigskip

Las condiciones de eliminacion se establecen sobre el parametro de
tratamiento/terapia o control y se determinan al aplicar la teoria de
estabilidad en el sentido de Lyanpunov, particularmente el metodo directo de
Lyapunov.

\bigskip

Se propone la siguiente funcion candidata de Lyapunov:

\bigskip

\begin{equation*}
V=x,
\end{equation*}

\bigskip

y se calcula su derivada

\bigskip

\begin{equation*}
V=\dot{x}=r_{1}x(1-b_{1}x)-a_{12}xy-a_{13}xz
\end{equation*}

\bigskip

y se reescribe la derivada de la siguiente forma

\bigskip

\begin{equation*}
V=\left( r_{1}-r_{1}b_{1}x-a_{12}-a_{13}z\right) x
\end{equation*}

\bigskip

ahora, al considerar los resultados del dominio de localizacion y evaluar la
derivada de este, es decir,

\bigskip

\begin{equation*}
\left. V\right\vert _{Kxyz}
\end{equation*}

\bigskip

se tiene lo siguiente:

\bigskip

\begin{equation*}
V=\left( r_{1}-a_{13}z_{\inf }\right) x\leq 0
\end{equation*}

a partir de esta expresion, se establece la siguiente condicion

\bigskip

\begin{equation*}
-a_{13}\frac{\rho i}{d_{3}}<0
\end{equation*}

\bigskip

por lo tanto, se despeja el parametro de tratamiento/terapia o control:

\bigskip

\begin{equation*}
\rho i>\frac{d_{3}r_{1}}{a_{13}}
\end{equation*}

\bigskip

y se establece el siguiente resultado

\bigskip

\textbf{Resultado IV: Condiciones de eliminacion.} Si la siguiente condicion
se cumple:

\bigskip

\begin{equation*}
\rho i>\frac{d_{3}r_{1}}{a_{13}}
\end{equation*}

\bigskip

\textit{entonces, se puede asegurar la eliminacion de la ponlacion descrita
por la variable }$x\left( t\right) $\textit{, es decir,}

\bigskip

\begin{equation*}
\lim_{t\rightarrow \infty }x\left( t\right) =0
\end{equation*}

\end{document}
